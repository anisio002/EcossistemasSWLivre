\section{Revisão da literatura e teoria}
% TODO: Resumir

\subsection{Metodologia de Desenvolvimento}
\label{desenvolvimento}

Uma questão central em Engenharia de Software consistem balancear as restrições conflitantes do projeto que incluem, mas não se limitam ao escopo, qualidade, tempo e custo .
Desenvolveu, ao longo do tempo, diferentes práticas e metodologias em busca de padrões que regem o desenvolvimento de software de qualidade observando dentro destas restrições. Em 2001, líderes técnicos criaram Aliança de Desenvolvimento Ágil e escreveram o Manisfesto Ágil, que pode ser resumido nas seguintes idéias:

\begin{itemize}
\item Indivíduos e interações acima de processos e ferramentas;
\item Software operante acima de documentações grandes e completas;
\item Colaboração do cliente acima de negociações contratuais;
\item Responder à mudanças acima de seguir a um planejamento.
\end{itemize}

Os métodos ágeis exaltam a simplicidade, feedback contínuo e adaptação à mudanças que podem ser obtidos 
a partir de comunicação face à face, qualidade de código e entrega contínua de software. Existem várias metodologias criadas a
partir deste conjunto de princípios, com destaque para o Scrum ~\cite{schwaber2001} e Extreme Programming (XP). Estes métodos 
não são excludentes e princípios e práticas de ambos podem ser utilizados de forma complementar ~\cite{fitzgerald2006}. 
Os métodos ágeis, especialmente quando aplicados a organizações maiores, importam inclusive elementos abordagens mais tradicionais 
com vistas a impor um maior rigor e previsibilidade no processo de criação de software.

\subsubsection{eXtreme Programming - XP}

O Extreme Programming - XP é uma metodologia de desenvolvimento de software com
foco em agilidade de equipes e qualidade de projetos, apoiado em cinco valores:
comunicação; feedback; simplicidade; coragem e respeito. O XP propõe, portanto,
a utilização de um conjunto de práticas conhecidas para serem aplicadas
disciplinadamente em busca do desenvolvimento de software de qualidade dentro
dos prazos e custos determinados.  

Assim como no Scrum, o XP define ciclos de desenvolvimento para que o produto de software seja incrementado a partir das prioridades do cliente. O XP introduziu uma série de técnicas e processos consagrados na indústria: testes automatizados com grande cobertura da base de código; programação em pares; o uso de histórias de usuários para definição de requisitos; o planning poker para estimar o esforço de implementação de cada funcionalidade ~\cite{cohn2005}; 

As práticas de XP, especialmente a ênfase no uso extensivo de testes automatizados, permitem que módulos construídos do software sejam integrados freqüentemente e com garantias razoáveis de que o software funciona de acordo com o esperado e que o ciclo de desenvolvimento não introduz novos defeitos. Isto facilita a entrega constante de software e interações rápidas com o cliente.

Do ponto de vista arquitetural, o XP encoraja o design simples que surge a partir da construção de cada funcionalidade e não a partir de atividades específicas de design que visam a implementação de uma arquitetura baseada em extensões futuras. O design surge naturalmente do processo de desenvolvimento aliado à prática de refatoração sistemática, com uma série de passos e técnicas que buscam construir, evolutivamente, a extensibilidade e manutenibilidade da arquitetura. 

\subsubsection{Iniciativas de Adoção de Métodos Ágeis na Administração Pública Federal}

As metodologias ágeis começaram a ganhar espaço no início da década de 2000,
regidas pelo Manifesto Ágil, e desde então vêm ganhando crescente popularidade.
No cenário mundial, elas já são metodologias bastante difundidas entre diversos
setores. Atualmente, diversas organizações públicas brasileiras estão iniciando
investimentos em adoção de contratações de fornecedores de software utilizando
métodos ágeis e, portanto, estão começando a difundir tais métodos também no
setor público~\cite{melo2012}.

Recentemente, o Tribunal de Contas da União publicou o Acórdão no 2314/2013 que contém 
um relatório de levantamento elaborado pela Secretaria de
Fiscalização de Tecnologia da Informação (SEFTI), cujo objetivo foi conhecer as
bases teóricas do processo de desenvolvimento de software com metodologia ágil,
além de conhecer experiências práticas de contratação realizadas por
instituições públicas federais. Nesse acórdão são relatadas as experiências de
adoção e contratação de serviços utilizando métodos ágeis no contexto do (i)
Tribunal Superior do Trabalho (TST); (ii) Banco Central do Brasil (Bacen);
(iii) Instituto do Patrimônio Histórico e Artístico Nacional (IPHAN); (iv)
Instituto Nacional de Estudos e Pesquisas Educacionais Anísio Teixeira (INEP),
e Supremo Tribunal Federal (STF) (TCU, 2013).

No referido acórdão, baseado nas experiências citadas, é concluído que, apesar
dos riscos inerentes em qualquer processo de adoção a adaptação de novas
metodologias na Administração Pública Federal (APF), a adoção de métodos ágeis
no âmbito da APF em contratos para desenvolvimento de software é
demonstradamente viável.

\subsection{Devops}
DevOps é contração entre development e operations. Trata-se da metodologia de desenvolvimento e entrega que mantêm os
times de desenvolvimento e infraestrutura em constante colaboração, muitas vezes borrando a fronteira entre estas duas etapas
~\cite{Httermann:2012:DD:2380958}. Isto é alcançado a partir de uma metodologia que enfatiza o uso de testes, integração contínua e automação 
para que, idealmente, os processos de implantação sejam tratados como código e que o código seja incorporado de forma contínua e 
confiável à aplicação em produção. 

A metodologia é normalmente associada a métodos ágeis de desenvolvimento, onde DevOps permite com que as mudanças realizadas 
pelo time desenvolvimento sejam incorporadas mais rapidamente em produção. Isto permite ciclos mais rápidos de atualização e uma 
maior facilidade em responder a novos requisitos e experimentar com novas funcionalidades. Várias companhias referência em desenvolvimento 
de software adotam práticas de DevOps, normalmente com resultados bastante positivos [Riungu-Kalliosaari et. al, 2016].

A cultura DevOps faz parte do conjunto de boas práticas de desenvolvimento preconizadas pelo LAPPIS. O uso desta metodologia no projeto 
permite não só uma maior agilidade no processo de desenvolvimento, mas possui um papel importante em capacitar os alunos com as tecnologias 
e processos utilizados no laboratório.

\subsection{Ecossistemas de Software Livre}

O desenvolvimento de software é semelhante a criação de uma receita, um
processo de aprendizado envolvendo tentativas e erros, o que difere
acentuadamente do passo-à-passo característico de se seguir uma
receita~\cite{poppendieck2011}. 

Entendendo software livre como um método que promove a participação desde o seu
processo de concepção e desenvolvimento, essa forma colaborativa de desenvolver
software é a mais adequada em um projeto que visa promover a participação
social, de forma aberta e transparente ao cidadã. 
Entre os aspectos em comum do universo open source, pode-se citar o fato de todas se valerem de
um modelo aberto e colaborativo, dinâmico e flexível, calcado na espontaneidade e voluntariedade.
Nesse meio, fazem-se presentes a cultura meritocrática e a
produção entre pares, elementos-chave da cultura hacker.

Aliados à cultura Devops, o ecossistema de software livre fomenta um arranho produtivo  formado por pessoas e organizações em torno de um bem comum de código,
que utilizam, mantém e aprimoram os softwares, normalmente com base nas suas próprias necessidades de uso. Muitas vezes essas necessidades 
e visões de futuro em relação aos softwares coincidem, gerando eficiência e abundância no esforço de desenvolvimento.

% Precisamente, os elementos
% do método que serão investigados neste projeto serão os seguintes:  Engenharia
% de Software Empírica~\cite{wohlin2000}, em Ambientes Governamentais.  A inovação neste projeto poderá ser
% observadda em 3 vetores:

% Arquitetura de uma plataforma livre, voltada ao desenvolvimento de
% softwares para o Ministério da Cultura e sociedade, que disporá de tecnologia de integração de
% serviços provendo suporte à: Sistemas de Versionamento; Sistema de
% Gerenciamento; Plataforma de Colaboração por meio de Redes Sociais Expressivas;
% Monitoramento de qualidade de código-fonte e Sistema de Indexação e Busca.

%  As comunidades de software livre são um tipo de arranjo produtivo formado por pessoas e organizações em torno de um bem comum de código 
% que utilizam, mantém e aprimoram os softwares, normalmente com base nas suas próprias necessidades de uso. Muitas vezes essas necessidades 
% e visões de futuro em relação aos softwares coincidem, gerando eficiência e abundância no esforço de desenvolvimento.
% 
% Em consonância a isso, o conceito de governo eletrônico tem sido atualizado em todo o mundo, buscando formas de incluir a sociedade civil na co-produção das tecnologias e políticas desenvolvidas no setor público. A estratégia de governança digital instituida pelo Ministério do Planejamento, estabelece que a utilização, pelo setor público, de tecnologias da informação e comunicação tem "o objetivo de melhorar a informação e a prestação de serviços visando incentivar a participação dos cidadãos, tornando o governo mais responsável, transparente e eficaz"\footnote{\url{https://www.governoeletronico.gov.br/egd}}.
% 
% Nesse contexto, é fundamental pesquisar e aprender com as comunidades de software livre que, tanto no Brasil quanto no mundo, o Estado participa por adesão e, a partir dos aprendizados com seus arranjos, orientar e capacitar os servidores e técnicos do MinC nas práticas de planejamento, gestão de softwares abertos, aprimorando os mecanismos de governança digital dos softwares presentes no portifólio do MinC.




\subsection{Aprendizado de Máquina}

Do ponto de vista de processamento de dados e ciência da informação, o projeto aborda duas questões importantes: 
processar o fluxo de dados de forma eficiente e incremental,
e a questão principal do projeto é o processamento de linguagem natural, tanto para guiar o proponente da lei Rouanet na plataforma SALIC
quanto o uso de chatbots para interagir com proponentes da Lei Rouanet.

Nessa pesquisa exploratória, usaremos bibliotecas de aprendizado de máquina e processamento de linguagem natural open source,
tais como scikit-learn \footnote{\url{https://github.com/scikit-learn/scikit-learn}} e 
tensor flow \footnote{\url{https://github.com/tensorflow/tensorflow}}, que atualmente são utilizadas tanto pela comunidade acadêmica quanto
pela indústria de software.  


\subsection{Dimensionalidade e Visualização de dados Culturais}
Uma etapa importante consiste em minerar os dados e indicadores culturais (big data) e disponibilizar essas informações 
 de forma simples para os usuários, gestores possam  compreender facilmente tais informações, facilitando a tomada de decisão e garantindo 
 acesso e transparência das informações. Do ponto de vista técnico, isto consiste projetar 
os pontos no espaço de features, que geralmente consiste em um número elevado de dimensões,  para o espaço 
bidimensional da tela de um computador. Trata-se de um problema muito bem estudado em machine learning  e que pode ser
abordado por uma grande quantidade de diferentes técnicas e algoritmos [Van Der Maaten, Postma, Van den Herik, 2009].

O desafio de pesquisa consiste em encontrar algoritmos que 1) possuam performance computacional aceitável em grandes volumes de dados, 2)
implementações que realizam atualização incremental e 3) produzam resultados de visualização consistentes com as expectativas de um usuário
leigo. A terceira área é a que merece o maior esforço de estudo já que as duas primeiras são muito bem estabelecidas. 


% \subsection{Do ateste das etapas e organização do fluxo de trabalho}
% TODO: melhorar o texto
% 
% A partir da fase de planejamento deste projeto, o planejamento geral aqui apresentado será continuamente refinado. Como 
% resultado deste refinamento serão definidos os ciclos de produção dos produtos/protótipos (releases), scripts de automaçã
% o resultantes das atividades de DevOps, algoritmos de mineração e de aprendizado de máquina, ou mesmo processos e documentação
% extritamente necessários e assosiados a essas ou outras atividades. Além dessas entregas, ocorrerá de forma transversal e colaborativa a capacitação e transferência de conhecimento às equipes do MinC, no sentido do aprimoramento da governança digital dos softwares.
% 
% 
% \textbf{Operação (produção)}
% 
% De acordo com o contínuo refinamento do planejamento, no que se refere à: produtos; protótipos; scripts de automação; algoritmos de mineração e tratamento de dados; algortimos de aprendizado de máquina a equipe CDT/FUB  liberará um resultado R[1,...,N] para uso externo, no ambiente de operação disponibilizado pela CGTEC/MinC.
% O credenciamento para liberação desses resultados (releases/algortimos/scripts) em ambiente de produção se dará: i) após o fim do período de homologação; ii) a marcação de aceitação das issues nos repositórios por parte das equipes do MinC e iii) ter recebido resposta aos ofícios de encaminhamento de encaminhamento das entregas produzidas pelas equipes UnB/CDT e recebidas pelas equipes do MinC/CGTEC.
% 
% \textbf{Homologação}
% 
% O período de ateste e homologação, por parte do MinC/CGTEC, se dará imediatamente após a disponibilização de uma release candidata RC-[1,...,N] no ambiente disponibilizado pela CGTEC-MinC. Decorrerá um período de 15-30 dias, a partir da disponibilização da release candidata, para que os últimos ajustes, correções de defeitos identificados, testes finais de migração dos dados, e correções de pequenos defeitos que por ventura, venham se materializar.
% 
% A linha de desenvolvimento da release candidata, em homologação, não terá atividades de desenvolvimento de novos requisitos. O objetivo é estabilizar a versão do resultado, com vistas à liberação para o ambiente de produção.
% 
% O credenciamento para liberação de um resultado canditado em ambiente de homologação se dará: i) após o fim do período de desenvolvimento de um ciclo, a consequente liberação dos resultados intermediários identificados por meio das issues no repositório; e iii) ter recebido o ateste provisório por parte do da equipe da SLTI/MP.
% 
% \textbf{Resultados Intermediários}
% 
% Mensalmente a equipe CDT/FUB disponibilizará uma realease intermediária, havendo acordo negocial e técnico, no ambiente de desenvolvimento e testes disponibilizado pela MinC/CGTEC. Trata-se de oportunidade da equipe da equipe do MinC/CGTEC poder realizar  de forma não controlada diferentes testes, retroalimentando o desenvolvimento da equipe CDT/FUB com feedbacks de possíveis alterações. Todas as alterações solicitadas por parte da equipe MinC/CGTEC serão tratadas como itens de backlog no repositório do projeto, e portanto, serão priozizados juntamente com as demais histórias de usuários.
% 
% O credenciamento para liberação de um resultado intermediário em ambiente de desenvolvimento e testes se dará: i) após o fim de dois ciclos de desenvolvimento (sprints) ou mesmo de features; ii) da integração do incremento de software; iii) da execução da suíte de testes integrado; iv) da formalização da entrega, por meio da execução da suíte do testes automatizados.
% 
% 
% \textbf{Ciclos de desenvolvimento}
% 
% A execução das etapas acontecerão preferencialmente nas dependências da UnB, não se limitando a este espaço, caso seja necessário. As sprints terão duração de 15 dias, representando a cadência do tempo em cada ciclo de desenvolvimento. Será medida a capacidade de produção da equipe CDT/FUB, de forma a se identificar o limitador do trabalho. A partir de então, a quantidade de trabalho possível de ser realizada dentro de cada ciclo da etapa será utilizada como parâmetro para que as equipes CDT/FUB e MinC/CGTEC possam utilizar nas reuniões de planejamento, que ocorrerão no primeiro dia de cada ciclo de desenvolvimento.
% 
% Ao final de cada ciclo, será realizada a reunião de revisão da sprint, onde a equipe CDT/FUB apresenta para a equipe MinC/CGTEC o incremento do protótipo construído no ciclo, caracterizando assim, o desenvolvimento guiado pelas prioridades elencadas pelo ministério. Ainda, ao final de duas sprints será gerado um protótipo intermediário e disponibilizada no ambiente de desenvolvimento e testes provido pela MinC/CGTEC.
% 
% A medida em que as atividades de pesquisa avançem e também as equipes CDT/FUB e MinC/CGTEC ganhem maturidade com as tecnologias de deenvolvimento e práticas de gestão, esse cadenciamento será evoluído para entregas contínuas, dentro dos ciclos de desenvolvimento dos incrementos dos produtos de software.
  

