% \section{Pesquisa}
% 
% %As referências não estão aparecendo.
% 
% O presente projeto também tem o objetivo de desenvolver pesquisa aplicada na
% área de Engenharia de Software. Em
% particular, a experiência no desenvolvimento de software livre e métodos ágeis, no contexto do
% Ministério da Cultura, com interface centrada no usuário, será utilizada para
% % fomentar a pesquisa e requerá um método específico. P
% %
% \begin{itemize}
% %
% \item A
% 
% \item Modelos de contratação de ambientes de desenvolvimento de software no contexto do Ministério da Cultura.
% %
% 
% \item Utilização de métodos ágeis e software livre para suportar a criação de políticas públicas
%


% \end{itemize}

% Processo de desenvolvimento é um elemento-chave na produção de software,
% podendo ser responsável direto pela qualidade do produto
% gerado~\cite{munch2012}. Deseja-se que tal processo leve a maior produtividade
% e compreensão dos aspectos intrínsecos à produção. Por outro lado, processos de
% desenvolvimento estão frequentemente sujeitos a restrições  de plataforma,
% como, por exemplo, no desenvolvimento de software embarcado crítico, sistemas
% de informação, entre outros. No contexto deste projeto, o requisito de se
% utilizar softwares livres, no contexto do software público, exige que seja
% feita uma customização de processos de desenvolvimento de software. Em
% particular, investigaremos como modelos de referência baseados em metodologias
% ágeis poderão ser customizadas para este projeto. Diferentemente de abordagens
% tradicionais tais como o Processo Unificado, as metodologias ágeis focam menos
% em modelos e enfatizam mais código, testes e colaboração entre pessoas, que são
% guiados por extratos incrementais do sistema englobando um conjunto de
% requisitos.  Neste contexto, um elemento importante da customização é a
% natureza inovadora do projeto. Esta natureza se apoia no escopo parcialmente
% definido e implicará em uma definição de estratégias de interação peculiares
% entre os desenvolvedores e pesquisadores.  Segundo a norma de
% qualidade~\cite{ISO25023},  o uso do produto software pode ser classificado e
% medido por meio de sua qualidade em uso, que influencia e depende de sua
% qualidade externa, que por sua vez, influencia e depende de sua qualidade
% interna. Visando a garantia de qualidade dos produtos a serem desenvolvidos, o
% projeto investigará o uso de técnicas modernas de validação e verificação, como
% por exemplo, testes automatizados, inspeções guiadas, e análises estáticas de
% código. Tais técnicas serão adaptadas para o contexto específico do projeto,
% considerando as diferentes plataformas a serem utilizadas, além do caráter  da
% inovação. Em nível de testes, serão explorados tanto testes unitários como
% teste de sistema e de usabilidade. Serão investigadas estratégias para alcançar
% o maior nível possível de automação de tais testes, a fim de diminuir o esforço
% dispendido na execução dos mesmos e, consequentemente, aumentar a produtividade
% do processo. Serão, adicionalmente, definidos testes de regressão com a
% finalidade de garantir a estabilidade das aplicações.  Com relação à validação
% estática, serão feitas inspeções guiadas de código, estudadas e aperfeiçoadas
% por técnicas de análise estática.  Em particular, a abordagem metodológica a
% ser empregada no projeto é prioritariamente de natureza empírica, qualitativa e
% de design. A natureza empírica decorre principalmente do foco do projeto na
% construção de soluções inovadoras aplicadas ao mundo real. Qualitativa porque
% as pesquisas envolvidas pretendem mergulhar em profundidade na compreensão dos
% fenômenos envolvidos no desenvolvimento da  plataforma, envolvendo não só ``o
% que'' ocorre, mas também como os fenômenos ocorrem, em termos da produção de
% software. Todavia, o projeto envolverá também a realização de estudos
% quantitativos para  descrever e avaliar outras técnicas e ferramentas
% empregadas, barreiras e facilitadores. Finalmente, a abordagem metodológica é
% também de design~\cite{hevner2004}, porque o projeto almeja produzir artefatos
% sob a forma de um guia metodológico. Abaixo apresentamos uma lista que resume a
% abordagem metodológica a ser seguida nas atividades de pesquisa do projeto:
% %
% \begin{itemize}
% %
% \item \textbf{Concepção  Filosófica:}   Pragmática~\cite{easterbrook2007}.
% \item \textbf{Método Científico:} Indutivo~\cite{marconi2004}.
% \item \textbf{Abordagem:} Qualitativa~\cite{monteiro2011,wohlin2000}.
% \item \textbf{Métodos de Pesquisa:} Revisão de literatura~\cite{kitchenham2009}; Pesquisa-Ação~\cite{merriam2009qualitative}; Estudo de Caso; Design Science~\cite{hevner2004}.
% \item \textbf{Procedimentos de Coleta de Dados:} Observação participante \cite{merriam2009qualitative,wohlin2000}.
% \item \textbf{Procedimentos de Análise:} Classificação temática e análise qualitativa de entrevistas, documentos e registros de observação participante~\cite{merriam2009qualitative}.
% %
% \end{itemize}
% 
% Em termos de fases da pesquisa, inicialmente, serão conduzidas revisões de
% literatura. Em seguida, serão definidos modelos de customização do processo de
% software contemplando as particularidades do projeto. Isto será feito usando-se
% uma abordagem visando a utilização sistemática de métodos ágeis de
% desenvolvimento de software. Nessa fase, o método de pesquisa usado será
% Pesquisa-Ação e estudos de caso, considerando-se todas as etapas do desenvolvimento.
% Posteriormente, será conduzido uma análise mais aprofundada dos resultados do
% estudo de forma a  estabelecer conclusões fundadas no método científico.
%  
